% Maths
\DeclareMathOperator{\reales}{\mathbb{R}}
\DeclareMathOperator{\enteros}{\mathbb{Z}}

\newcommand{\sumatoria}[2]{\sum\limits_{#1}^{#2}}
\newcommand{\productoria}[2]{\prod\limits_{#1}^{#2}}

\DeclareMathOperator{\vacio}{\varnothing}
\DeclareMathOperator{\distinto}{\neq}
\renewcommand{\mod}{\bmod}
\DeclareMathOperator{\en}{\in}
\DeclareMathOperator{\existe}{\exists}
\DeclareMathOperator{\paratodo}{\forall}
\DeclareMathOperator{\entonces}{\rightarrow}
\DeclareMathOperator{\sii}{\leftrightarrow}
\DeclareMathOperator{\orL}{\,\lor_{\text{L}}\,}
\DeclareMathOperator{\andL}{\,\land_{\text{L}}\,}
\DeclareMathOperator{\entoncesL}{\,\rightarrow_{\text{L}}\,}
\DeclareMathOperator{\siiL}{\leftrightarrow_{\text{L}}}
\def\falso{\texttt{\green{falso}}\xspace}
\def\verdadero{\texttt{\blue{verdadero}}\xspace}

\newcommand{\ub}[2]{ \underbrace{\textstyle #1}_{\mathclap{#2}} }
\newcommand{\ob}[2]{ \overbrace{\textstyle #1}^{\mathclap{#2}} }
\newcommand{\secuencia}[1]{\texttt{seq}\langle #1 \rangle}
\newcommand{\kets}[1]{\langle #1 \rangle}

\def\en{\in}

% comandos de pseudocodigo
\def\IfThenElseFi{\texttt{IfThenElseFi}}
\def\pertenece{\texttt{pertenece}}

%% ESCRIBIR CODIGO
% Nuevos nombres para las keywords
\SetKwInput{requiere}{requiere}
\SetKwInput{asegura}{asegura}

% formats
\SetCustomAlgoRuledWidth{\textwidth}
\SetAlgoCaptionLayout{centerline}
\SetAlFnt{\small \ttfamily}
\SetAlgoNlRelativeSize{0}

% comments
\newcommand{\commentstyle}[1]{\textcolor{OliveGreen}{\ttfamily\itshape #1}}
\SetCommentSty{commentstyle}
\SetKwComment{comment}{//}{}

% Line numbers
\SetNlSty{tt}{\color{purple}}{}

%% FIN ESCRIBIR CODIFGO

% separadores
\newcommand{\separador}{
  \par\noindent\rule{\linewidth}{0.4pt}\par
}
\newcommand{\separadorCorto}{
  \par\noindent\rule{0.5\linewidth}{0.4pt}\par
}

% Colores
\newcommand{\red}[1]{\textcolor{red}{#1}}
\newcommand{\green}[1]{\textcolor{OliveGreen}{#1}}
\newcommand{\blue}[1]{\textcolor{Cerulean}{#1}}
\newcommand{\cyan}[1]{\textcolor{cyan}{#1}}
\newcommand{\yellow}[1]{\textcolor{YellowOrange}{#1}}
\newcommand{\magenta}[1]{\textcolor{magenta}{#1}}
\newcommand{\purple}[1]{\textcolor{purple}{#1}}
\newcommand{\rosa}[1]{\textcolor{pink}{#1}}
\newcommand{\transparente}[1]{\color[rgb]{1,1,1,0.5}{#1}}
\newcommand{\marron}[1]{\color{brown}{#1}}

%% Cartelito HACER EJERCICIO
\newcommand{\hacer}{
  {\color{red!80!black}{\faIcon[regular]{flushed}... hay que hacerlo! \faIcon[regular]{sad-cry}}}\par
  {\color{black!70!white}
    \small Si querés mandarlo: Telegram $\to$ \href{\dirTelegram}{\small\faIcon{telegram}},
    o  mejor aún si querés subirlo en \LaTeX $\to$ \href{\dirRepo}{\small \faIcon{github}}.
  }\par
}

% Update time
\def\update{\tiny
  {\today\ @ \currenttime}
}

%=======================================================
% sección ejercicio con su respectivo formato y contador
%=======================================================
%iconito
\def\fueguito{{\color{orange}{\faIcon{fire}}}}

\newcounter{ejercicio}[section] % contador que se resetea en cada sección
\renewcommand{\theejercicio}{\arabic{ejercicio}} % el contador es un número arabic
\newcommand{\ejercicio}{%
  \stepcounter{ejercicio}% incremento en uno
  \titleformat{\section}[runin]{\bfseries}{\theejercicio}{1em}{}%
  \section*{Ejercicio \theejercicio}\labelEjercicio{ej:\theejercicio}
}

% Label y refencia para ejercicio hay alguna forma más elegante de hacer esto?
\newcommand{\labelEjercicio}[1]{
  \addtocounter{ejercicio}{-1} % counter - 1
  \refstepcounter{ejercicio} % referencia al anterior y luego + 1
  \label{#1}
}
\newcommand{\refEjercicio}[1]{{\bf \ref{#1}.}}

%%%=======================================================
%%% fin sección ejercicio con su respectivo formato y contador
%%%=======================================================

% Enviroment ENUNCIADo
\newenvironment{enunciado}[1]{% Toma un parametro obligatorio: \ejExtra o \ejercicio 
  \par
  \noindent
  \begin{minipage}{\linewidth}
    \separador % linea sobre el enunciado
    #1
    }% contenido
    {
    \separadorCorto % linea debajo del enunciado
    \par
  \end{minipage}\par
}
%% Predicado
\newcommand{\predicado}[3]{
  \texttt{pred \purple{#1}} \textrm{$\bigl($#2$\bigr)$ $\bigl\{$} \par
  \hspace{2em} #3 \par
  \textrm{$\bigr\}$}
}

%% Procedimiento
\newcommand{\procedimiento}[5]{
  \texttt{proc #1} $\bigl($#2$\bigr)$ : #3 \par
  \hspace{2em}\texttt{requiere}\{ #4 \} \par
  \hspace{2em}\texttt{asegura}\{ #5 \}
}

%% Auxiliares
\newcommand{\auxiliar}[4][]{
        \texttt{aux \purple{#2}} $\bigl($#3$\bigr)$#1 $=$ #4
}

\def\True{\texttt{True}\xspace}
\def\False{\texttt{False}\xspace}
\def\res{\texttt{res}\xspace}
\def\Bool{\texttt{Bool}\xspace}



%=======================================================
% Comandos con flechas extensibles.
%=======================================================
% *Flechita* extensible con texto {arriba} y [abajo] 
\NewDocumentCommand{\flecha}{m o}{%
  \IfNoValueTF{#2}{%
    \xrightarrow[]{\text{#1}}
  }{
    \xrightarrow[\text{#2}]{\text{#1}}
  }
}
% *Si solo si* extensible con texto {arriba} y [abajo] 
\NewDocumentCommand{\Sii}{m o}{%
  \IfNoValueTF{#2}{%
    \xLeftrightarrow[]{\text{#1}}
  }{
    \xLeftrightarrow[\text{#2}]{\text{#1}}
  }
}

% *Si solo si* extensible con texto {arriba} y [abajo] 
\NewDocumentCommand{\Entonces}{m o}{%
  \IfNoValueTF{#2}{%
    \xRightarrow[]{\text{#1}}
  }{
    \xRightarrow[\text{#2}]{\text{#1}}
  }
}

%=======================================================
% fin comandos con flechas extensibles.

% como el stackrel pero también se puede poner algo debajo
\newcommand{\taa}[3]{ % [t]exto [a]rriba y [a]bajo
  \overset{\mathclap{#1}}{\underset{\mathclap{#2}}{#3}}
}

% Contributors
\newcommand{\aporte}[2]{
  \href{#1}{#2}
}

\newenvironment{aportes}{
  \par
  \vspace{10pt}
  \begin{minipage}{1\linewidth}
    \tt\footnotesize
    Los culpables de que esto haya sucedido:
    \vspace{-10pt}
    \begin{multicols}{3}
      \begin{itemize}[label={\tiny\yellow{\faIcon{medal}}}]
        }{
      \end{itemize}
    \end{multicols}
  \end{minipage}
}

%% links
\def\libre{https://youtu.be/7812dngARbk?t=43}

%%% Iconos más usados
\def\github{\faIcon{github}}
\def\instagram{\faIcon{instagram}}
\def\tiktok{\faIcon{tiktok}}
\def\linkedin{\faIcon{linkedin}}
