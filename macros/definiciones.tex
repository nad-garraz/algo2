% Maths
\DeclareMathOperator{\reales}{\mathbb{R}}
\DeclareMathOperator{\enteros}{\mathbb{Z}}

\DeclareMathOperator{\distinto}{\neq}
\renewcommand{\mod}{\bmod}
\DeclareMathOperator{\existe}{\exists}
\DeclareMathOperator{\paratodo}{\forall}
\DeclareMathOperator{\entonces}{\rightarrow}
\DeclareMathOperator{\sii}{\leftrightarrow}
\DeclareMathOperator{\orL}{\lor_{\text{L}}}
\DeclareMathOperator{\andL}{\land_{\text{L}}}
\DeclareMathOperator{\entoncesL}{\rightarrow_{\text{L}}}
\DeclareMathOperator{\siiL}{\leftrightarrow_{\text{L}}}
\def\falso{\textit{falso}\xspace}
\def\verdadero{\textit{verdadero}\xspace}

\newcommand{\ub}[2]{ \underbrace{\textstyle #1}_{\mathclap{#2}} }
\newcommand{\ob}[2]{ \overbrace{\textstyle #1}^{\mathclap{#2}} }

\def\en{\in}

%% ESCRIBIR CODIFGO
% Nuevos nombres para las keywords
\SetKwInput{requiere}{requiere}
\SetKwInput{asegura}{asegura}

% formats
\SetCustomAlgoRuledWidth{\textwidth}
\SetAlgoCaptionLayout{centerline}
\SetAlFnt{\small \ttfamily}
\SetAlgoNlRelativeSize{0}

% comments
\newcommand{\commentstyle}[1]{\textcolor{OliveGreen}{\ttfamily\itshape #1}}
\SetCommentSty{commentstyle}
\SetKwComment{comment}{//}{}

% Line numbers
\SetNlSty{tt}{\color{purple}}{}

%% FIN ESCRIBIR CODIFGO

% separadores
\newcommand{\separador}{
  \par\noindent\rule{\linewidth}{0.4pt}\par
}
\newcommand{\separadorCorto}{
  \par\noindent\rule{0.5\linewidth}{0.4pt}\par
}

% Colores
\newcommand{\red}[1]{\textcolor{red}{#1}}
\newcommand{\green}[1]{\textcolor{OliveGreen}{#1}}
\newcommand{\blue}[1]{\textcolor{Cerulean}{#1}}
\newcommand{\cyan}[1]{\textcolor{cyan}{#1}}
\newcommand{\yellow}[1]{\textcolor{YellowOrange}{#1}}
\newcommand{\magenta}[1]{\textcolor{magenta}{#1}}
\newcommand{\purple}[1]{\textcolor{purple}{#1}}
\newcommand{\rosa}[1]{\textcolor{pink}{#1}}
\newcommand{\transparente}[1]{\color[rgb]{1,1,1,0.5}{#1}}
\newcommand{\marron}[1]{\color{brown}{#1}}

%% Cartelito HACER EJERCICIO
\newcommand{\hacer}{
  {\color{red!80!black}{\faIcon[regular]{flushed}... hay que hacerlo! \faIcon[regular]{sad-cry}}}\par
  {\color{black!70!white}
    \small Si querés mandarlo: Telegram $\to$ \href{\dirTelegram}{\small\faIcon{telegram}},
    o  mejor aún si querés subirlo en \LaTeX $\to$ \href{\dirRepo}{\small \faIcon{github}}.
  }\par
}

% Update time
\def\update{\tiny
  {\today\ @ \currenttime}
}

%=======================================================
% sección ejercicio con su respectivo formato y contador
%=======================================================
%iconito
\def\fueguito{{\color{orange}{\faIcon{fire}}}}

\newcounter{ejercicio}[section] % contador que se resetea en cada sección
\renewcommand{\theejercicio}{\arabic{ejercicio}} % el contador es un número arabic
\newcommand{\ejercicio}{%
  \stepcounter{ejercicio}% incremento en uno
  \titleformat{\section}[runin]{\bfseries}{\theejercicio}{1em}{}%
  \section*{Ejercicio \theejercicio}\labelEjercicio{ej:\theejercicio}
}

% Label y refencia para ejercicio hay alguna forma más elegante de hacer esto?
\newcommand{\labelEjercicio}[1]{
  \addtocounter{ejercicio}{-1} % counter - 1
  \refstepcounter{ejercicio} % referencia al anterior y luego + 1
  \label{#1}
}
\newcommand{\refEjercicio}[1]{{\bf \ref{#1}.}}

%%%=======================================================
%%% fin sección ejercicio con su respectivo formato y contador
%%%=======================================================

% Enviroment ENUNCIADo
\newenvironment{enunciado}[1]{% Toma un parametro obligatorio: \ejExtra o \ejercicio 
  \par
  \noindent
  \begin{minipage}{\linewidth}
    \separador % linea sobre el enunciado
    #1
    }% contenido
    {
    \separadorCorto % linea debajo del enunciado
    \par
  \end{minipage}\par
}
\def\libre{https://youtu.be/7812dngARbk?t=43}
