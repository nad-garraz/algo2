\begin{enunciado}{\ejercicio}
  Considerar las siguientes dos especificaciones, junto con un algoritmo $a$
  que satisface la especificación de \texttt{p2}.\par
  \procedimiento{p1}{in $x: \reales$, in $n: \enteros$}
  {$\enteros$}
  {$x \distinto 0$}
  {$x^n - 1 < \res \leq x^n$}

  \procedimiento{p2}{in $x: \reales$, in $n: \enteros$}
  {$\enteros$}
  {$n\leq 0 \entonces x \distinto 0$}
  {$\res = \lfloor x^n \rfloor$}

  \begin{enumerate}[label=\alph*)]
    \item Dados valores de $x$ y $n$ qeu hacen verdadera la precondición de \texttt{p1}, demostrar que hacen también verdadera
          la precondición de \texttt{p2}.

    \item Ahora, dados estos valores de $x$ y $n$, supongamos que se ejecuta $a$: Llegamos a un valor de \res que hace
          verdadera la postcondición de \texttt{p2} ¿Será también verdadera la postcondición de \texttt{p1} con este valor de \res?

    \item ¿Podemos concluir que $a$ satisface la especificación de \texttt{p1}?
  \end{enumerate}
\end{enunciado}

\begin{enumerate}[label=\alph*)]
  \item En una implicación si el \textit{consecuente} es \verdadero, no importa el valor del \textit{antecedente}, la implicación será siempre \verdadero. $n \leq0 \entonces x \distinto 0$
        \begin{center}
          \begin{tabular}{|c|c|c|}
            \hline
            $n\leq 0$ & $x\distinto 0$ & $n \leq 0 \entonces x \distinto 0$ \\\hline
            1         & 1              & 1                                  \\
            0         & 1              & 1                                  \\\hline
          \end{tabular}
        \end{center}

  \item La función cumple:
        $$
          y \en \reales \entonces \lfloor y \rfloor \leq y
        $$
        Y si le resto 1 a $y$ cambia el número de la unidad así que:
        $$
          y - 1 \en \reales \entonces y - 1 \leq \lfloor y \rfloor
        $$

  \item Sé que el algoritmo $a$ cumple \texttt{p2}. Luego, los \texttt{requiere} son equivalentes, y \texttt{asegura2} $\taa{\red{??}}{}\subseteq$ \texttt{asegura1}. Entonces debería satisfacer,
        porque \texttt{p1} sería una subespecificación de \texttt{p2}.\par
        \red{RECONTRA CONTROLAR!!!}
\end{enumerate}
