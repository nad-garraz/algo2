\begin{enunciado}{\ejercicio}
  Especificar los siguietnes problemas sobre secuencias:
  \begin{enumerate}[label=\alph*)]
    \item Dadas dos secuencias $s$ y $t$, decidir si $s$ está \textit{incluida} en $t$, es decir,
          si todos los elementos de $s$ aparecen en $t$ en igual o mayor cantidad.
    \item

    \item Dada una secuencia de números enteros, devolver aquel que divida a más elementos
          de la secuencia. El elemento tiene que pertenecer a la secuencia original.
          Si existe más de un elemento que cumple esta propiedad, devolver alguno de ellos.
    \item
    \item
  \end{enumerate}
\end{enunciado}

\begin{enumerate}[label=\alph*)]
  \item
        \procedimiento{esSubconjunto}
        {s: $\enteros$, t:$\enteros$}
        {\Bool}
        {\True}
        {$(\paratodo y \en s) \entoncesL \pertenece(t,y)$}

        \procedimiento{esSubconjunto2}
        {s: $\enteros$, t:$\enteros$}
        {\Bool}
        {\True}
        {$(\paratodo y :\enteros)
            \bigl(
            \pertenece(s,y) \entoncesL \pertenece(t,y)
            \bigr)$}

        Por si acaso: \hyperlink{aux:pertenece}{definición de auxiliar \pertenece}

        \red{Consultar por sintaxis de los asegura. También no se especifica el \texttt{tipo}
          cómo se pone el genérico? T?}

  \item

  \item
        \procedimiento{cantElemQDiv}
        {in $s: \secuencia{\enteros}, e: \enteros$} {$\enteros$}
        {$e \distinto 0$}
        {$\res = \sumatoria{i=0}{|s| - 1} \IfThenElseFi(s[i] \mod e = 0, 1, 0)$}\par\bigskip

        \procedimiento{elMasDivisor}
        {in $s: \secuencia{\enteros}$} {$\enteros$}
        {$|s| > 0$}
        {$\scriptstyle
            (y \en s) \land (y \distinto 0) \land
            \bigl(
            \res = y \sii
            (\paratodo y' \en s)
            ( (y'\distinto 0) \entoncesL \texttt{cantElemQDiv}(s, y) \geq \texttt{cantElemQDiv}(s, y'))
            \bigr)$ }

        Cuando estoy escribiendo el asegura. Yo tengo que contemplar con que el $\res = y$ sea

  \item
  \item
\end{enumerate}
