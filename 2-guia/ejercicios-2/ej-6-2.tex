\texttt{Análisis de especificación}
\begin{enunciado}{\ejercicio}
  Las siguientes especificaciones no son correctas. Indicar por qué y
  corregirlas para que describan correctamente el problema.
  \begin{enumerate}[label=\alph*)]
    \item \texttt{progresionGeometricaFactor2}: Indica si la secuencia $l$ representa una prograsión
          geométrica factor 2. Es decir, si cada elemento de la secuencia es el doble del elemento
          anterior.

          \procedimiento{progresionGeometricaFactor2}
          {in $l: \seq{\enteros}$}
          {\Bool}
          {\True}
          {$\res = \True \sii ((\paratodo i : \enteros) \bigl(0 \leq i < |l| \entoncesL l[i] = 2 \cdot l[i-1])\bigr)$}

    \item \texttt{minimo:} Devuelve en \res el menor elemento de $l$.

          \procedimiento{minimo}
          {in $l:\seq{\enteros}$}
          {$:\enteros$}
          {\True}
          {$(\paratodo y : \enteros)\bigl( (y \en l \land y \distinto x) \entonces y > \res \bigr)$}

  \end{enumerate}
\end{enunciado}

\begin{enumerate}[label=\alph*)]
  \item Caso borde cuando el índice es \red{i = 0}, estaría intentando acceder a \red{$l[-1]$}

        \procedimiento{progresionGeometricaFactor2}
        {in $l: \seq{\enteros}$}
        {\Bool}
        {\True}
        {$\res = \True \sii ((\paratodo i : \enteros) \bigl(0 \taa{\red{!}}{}< i < |l| \entoncesL l[i] = 2 \cdot l[i-1])\bigr)$}

  \item Caso borde cuando la lista tiene todos los números iguales, o un solo número. Y también
        está esa \red{$x$} ahí medio perdida. La respuesta debe estar contenida en la lista, sino se podría poner cualquier
                número que sea menor a todos los elementos de la lista.\par
        \procedimiento{minimo}
        {in $l:\seq{\enteros}$}
        {$\enteros$}
        {\purple{$|l| > 0$}}
        {$\res \en l \land (\paratodo y : \enteros)\bigl( (y \en l \land y \distinto \purple{\res}) \entonces y \taa{\red{!}}{}\geq \res \bigr)$}
\end{enumerate}
