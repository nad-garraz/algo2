\begin{enunciado}{\ejercicio}
  Sean $x$ y $r$ variables de tipo $\reales$. Considerar los siguientes predicados:
  \begin{multicols}{2}
    P1: $\{x \leq 0\}$\par
    P2: $\{x \leq 10\}$\par
    P3: $\{x \leq -10\}$\par
    Q1: $\{r \geq x^2\}$\par
    Q2: $\{r \geq 0\}$\par
    Q3: $\{r = x^2\}$\par
  \end{multicols}

  \begin{enumerate}[label=\alph*)]
    \item Indicar la relación de fuerza entre P1, P2 y P3.
    \item Indicar la relación de fuerza entre Q1, Q2 y Q3.
    \item Escribir 2 programas que cumplan con la siguietne especificación:\par
          \procedimiento{hagoAlgo}
          {in $x:\reales$}
          {$\reales$}
          {$x \leq 0$}
          {$\res \geq x^2$}
    \item \hacer
    \item ¿Qué conclusión pueden sacar? ¿Qué debe cumplirse con respecto a las precondiciones y postcondiviones para que sea seguro
          \textbf{reemplazar la especificación}?
  \end{enumerate}
\end{enunciado}

Dadas las proposiciones lógicas $\alpha$ y $\beta$, se dice que $\alpha$ es más fuerte que
$\beta$ si y solo si $\alpha \entonces \beta$ es una tautología.
En este caso, también decimos que $\beta$ es más debil que $\alpha$.

\begin{enumerate}[label=\alph*)]
  \item Indicar la relación de fuerza entre P1, P2 y P3.
        \begin{itemize}
          \item Me gusta pensar que $P1 \subseteq P2$, por lo cual P1 debería ser más fuerte que P2.\par
                \begin{center}
                  \begin{tabular}{|c|c|c|c|c|c|}
                    \hline
                    ej       & P1 & P2 & P1 $\entonces$ P2 & P2 $\entonces$ P1 \\\rowcolor{lightgray}\hline
                    $x=-1$   & 1  & 1  & 1                 & 1                 \\\rowcolor{lightgray}
                    $x=5$    & 0  & 1  & 1                 & \red{0}           \\
                    $\vacio$ & 1  & 0  & -                 & -                 \\\rowcolor{lightgray}
                    $x=11$   & 0  & 0  & 1                 & 1                 \\\hline
                  \end{tabular}
                \end{center}
          \item Siguiendo el razonamiento P3 $\subseteq$ P1 $\subseteq$ P2\par
                La proposición más fuerte va a ser P3.
        \end{itemize}

  \item Siguiendo el razonamiento del inciso anterior: Q3 $\subseteq$ Q2 $\subseteq$ Q1.
        La más fuerte es Q3.

  \item \red{ Programa en JAVA?}

  \item

  \item ¿Qué conclusión pueden sacar? ¿Qué debe cumplirse con respecto a las precondiciones y postcondiviones para que sea seguro
        \textbf{reemplazar la especificación}?
\end{enumerate}
