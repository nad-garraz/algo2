  \begin{enunciado}{\ejercicio}
    Sean $p, q$ y $r$ tres variables de las que se sabe que:
    \begin{enumerate}[label=\tiny$\blacksquare$]
      \item $p$ y $q$ nunca están indefinidas
      \item $r$ se indefine sii $q$ es \textit{verdadera}
    \end{enumerate}
    Proponer una fórmula que nunca se indefina, utilizando siempre las tres variables y que sea verdadera si y solo si se cumple que:
    \begin{multicols}{2}
      \begin{enumerate}[label=\alph*)]
        \item Al menos una es verdadera.
        \item Ninguna es verdadera.
        \item Exactamente una de las tres es verdadera.
        \item Solo $p$ y $q$ son verdaderas.
        \item No todas al mismo tiempo son verdaderas.
        \item $r$ es verdadera.
      \end{enumerate}
    \end{multicols}
  \end{enunciado}

  \begin{enumerate}[label=\alph*)]
    \item Al menos una es verdadera.
          $$
            (p \lor q) \orL r
          $$
          \red{Preguntar por solución del apunte, pqr=100?}

    \item Ninguna es verdadera. En este caso $r$ no se indefine.
          $$
            \neg(p \land q \land r)
          $$

          \red{Preguntar por solución del apunte, pqr=100?}

    \item Exactamente una de las tres es verdadera.
          $$
            (p \lor q) \orL r
          $$

    \item Solo $p$ y $q$ son verdaderas. $r$ debe ser 0 cuando $q = 0$
          $$
            (p \lor q) \orL \neg r
          $$
          \red{solo $p$ y $q$?}

    \item No todas al mismo tiempo son verdaderas.

    \item $r$ es verdadera.
          \red{Preguntar por solución del apunte, pqr=100?}

  \end{enumerate}
