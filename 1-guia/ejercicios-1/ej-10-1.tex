\begin{enunciado}{\ejercicio}
  Sean $P(x:\enteros)$ y $Q(x: \enteros)$ dos predicados que nunca se indefinen. Escribir al predicado asociado
  a cada uno de los siguientes enunciados:
  \begin{enumerate}[label=\tiny\faIcon{yin-yang}$_{\arabic*}$]
    \item \textit{"Existe un único natural menor a 10 que cumple P"}
    \item \textit{"Existen al menos dos números naturales menores a 10 que cumplen P"}
    \item \textit{"Existen exactamente dos números naturales menores a 10 que cumplen P"}
    \item \textit{"Todos los enteros pares que cumplen P, no cumplen Q"}
    \item \textit{"Si un entero cumple P y es impar, no cumple Q"}
    \item \textit{"Todos los enteros pares cumplen P, y todos los enteros impares que no cumplen P cumplen Q"}
    \item \textit{"Si hay un número natural menor a 10 que no cumple P entonces ninfuno natural menor a 10 cumple Q; y si todos los
            naturales menores a 10 cump,en P entonces hay al menos dos naturales menores a 10 que cumplen Q"}
  \end{enumerate}
\end{enunciado}

\begin{enumerate}[label=\tiny\faIcon{yin-yang}$_{\arabic*}$]
  \item \textit{"Existe un único natural menor a 10 que cumple P"}\par
        $$
          (\existe n : \enteros)
          \biggl(
          \bigl(
          0 \leq n < 10 \land P(n)
          \bigr)
          \land
          (\paratodo m : \enteros) \bigl(0 \leq m < 10  \entonces (m = n \entonces P(m))  \bigr) \biggr)
        $$

  \item \textit{"Existen al menos dos números naturales menores a 10 que cumplen P"}

        $$
          [\existe n: \enteros]
          \ob{
            \ub{\bigl[ 0 \leq n < 10 \land P(n) \bigr]}{f_1(n)}
            \land
            [\existe m: \enteros]
            \ub{\bigl[ 0 \leq m < 10 \land P(m) \land m \distinto n \bigr]}{f_2(n,m)}
          }{f_3}
        $$
        \begin{enumerate}[label=\tiny\faIcon{meh-rolling-eyes}]
          \item El \textit{cuantificador existencial} me generaliza la disyunción:
                $$
                  \dots
                  \lor f_1(-N)
                  \lor f_1(-N+1)
                  \lor\dots
                  \lor \blue{f_1(0)}
                  \lor\dots
                  \lor \blue{f_1(9)}
                  \lor\dots
                  \lor f_1(N-1)
                  \lor f_1(N)
                  \lor\dots
                $$
                Los \blue{$f_1(n)$ azules,} tienen valores de \verdadero, para la primera condición de $f_1$, los demás son falsos.
                Como tengo concatenación de disyunciones, con que una de las \blue{azules} cumpla también $P(n)$, listo tengo un $f_1$
                \verdadero para algún $n$.
                $$
                  \blue{f_1(0)}
                  \lor\dots
                  \lor \blue{f_1(9)}
                $$
                Esos valores son los que me sirven para calcular los valores de verdad que hay en $f_1 \land f_2$.

          \item Muy parecido ahora con $f_2$
                $$
                  \dots
                  \lor f_2(-N, \blue{n})
                  \lor f_2(-N+1, \blue{n})
                  \lor\dots
                  \lor \purple{f_2(0, \blue{n})}
                  \lor\dots
                  \lor \purple{f_2(9, \blue{n})}
                  \lor\dots
                  \lor f_2(N-1, \blue{n})
                  \lor f_2(N, \blue{n})
                  \lor\dots
                $$
                Nuevamente alguno de los \purple{$m$}, van a cumplir las condiciones \purple{$f_1(m,n)$}

          \item Por lo tanto obtengo siempre valores verdaderos en la expresión final $f_3$.
        \end{enumerate}

  \item \textit{"Existen exactamente dos números naturales menores a 10 que cumplen P"}

  \item \textit{"Todos los enteros pares que cumplen P, no cumplen Q"}

  \item \textit{"Si un entero cumple P y es impar, no cumple Q"}
        $$
          (\paratodo n: \enteros)\bigr( (P(n) \land n \mod 2 \distinto 0 ) \entonces \neg Q(n) \bigr)
        $$

  \item \textit{"Todos los enteros pares cumplen P, y todos los enteros impares que no cumplen P cumplen Q"}

  \item \textit{"Si hay un número natural menor a 10 que no cumple P entonces ninfuno natural menor a 10 cumple Q; y si todos los
          naturales menores a 10 cump,en P entonces hay al menos dos naturales menores a 10 que cumplen Q"}
\end{enumerate}
