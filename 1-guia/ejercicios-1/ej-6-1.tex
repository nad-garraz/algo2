\begin{enunciado}{\ejercicio}
  Asumiendo que el valor de $b$ y $c$ es verdadero, el de $a$ es $falso$ y el de $x$ e $y$ es indefinido, indicar cuáles de los operadores
  deben ser operadores "luego" para que la expresión no se indefina nunca:\par
  \begin{enumerate}[label=\alph*)]
    \item $(\neg x \lor b)$
    \item $((c \lor (y \land a)) \lor b)$
    \item
    \item
    \item
    \item $(((c \lor y) \land (a \lor b)) \sii (c \lor (y \land a) \lor b))$
    \item
  \end{enumerate}
\end{enunciado}

\begin{enumerate}[label=\alph*)]
  \item $(\neg x \lor b)$.\par
        Tengo que evaluar de \textit{izquierda a derecha} y se cortocircuita si el resultado de la fórmula es independiente de lo que resta leer.
        Cuando hay un disyunción con tener $x = F$, de manera que $\neg x = T$ ya puedo parar de evaluar. Pero si $x$ está indefinida, no hay nada que
        peuda hacer.

  \item $((c \lor (y \land a)) \lor b)$
  \item
  \item
  \item
  \item
        Para que la disyunción seguro no se indefina, necesito que el primer predicado sea \textit{verdadero} y la conjunción eso sucede si el primer predicado
        es \textit{falso}.
        $$
          ((\ub{(c \lor y)}{\substack{c \to \verdadero \\ \text{No se indefine}}}
          \land
          \ob{(a \lor b)}{\verdadero})
          \sii
          (\ob{c \lor \ub{(y \land a)}{\perp}}{\text{\red{¡necesito }}\orL \red{!}} \lor b))
        $$
  \item
\end{enumerate}
