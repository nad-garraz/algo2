\begin{enunciado}{\ejercicio}
  Sea $P(x: \enteros)$ y $Q(x: \enteros)$ dos predicados cualquiera. Explicar cuál es el error de traducción a
  fórmulas de los siguientes enunciados. Dar un ejemplo en el cuál sucede el problema y luego corregirlo.
  \begin{enumerate}[label=\alph*)]
    \item "\textit{Todos los naturales menores a 10 cumplen P}":
          $$
            (\paratodo i : \enteros)((0 \leq i < 10) \land P(i))
          $$

    \item "\textit{Algún natural menor a 10 cumple P}":
          $$
            (\existe i : \enteros)((0 \leq i < 10) \entonces P(i))
          $$

    \item "\textit{Todos los naturales menores a 10 que cumplen P y Q}":
          $$
            (\paratodo x : \enteros)((0 \leq x < 10) \entonces (P(x) \land Q(x)))
          $$

    \item "\textit{No hay ningún natural menor a 10 que cumpla $P$ y $Q$}":
          $$
            (\paratodo x : \enteros)((0 \leq x < 10) \entonces (P(x) \land Q(x)))
          $$
  \end{enumerate}
\end{enunciado}

\begin{enumerate}[label=\alph*)]
  \item "\textit{Todos los naturales menores a 10 cumplen P}":
        $$
          (\paratodo i : \enteros)\ub{(\ob{(0 \leq i < 10)}{f(i)} \land P(i))}{R(i)} \sii (\paratodo i : \enteros)(R(i))
        $$
        Como el cuantificador $\red{\paratodo}$ generaliza la conjunción. Suponiendo que $P(i)$ \red{no se indefine}:
        $$
          (\paratodo i : \enteros)(R(i))
          \sii
          R(-N) \land R(-N+1)
          \land \dots \land
          \blue{R(0)}
          \land \blue{\dots} \land
          \blue{R(9)} \land
          R(10)
          \land \dots \land
          R(N)
        $$
        Obtengo un resultado donde sé que los $\blue{R(0)},\blue{\dots}, \blue{R(9)}$ van a tener un valor \verdadero, pero el resto de los $R(i)$ van a tener
        un valor de \falso, ya que no cumplen $f(i)$.\par
        Por lo tanto busco una solución para que esos $R(i)$ no sean moscas en mi sopa, con "\red{$\entonces$}" dado que:
        $$
          \ub{\falso \entonces \text{\marron{\faIcon{poo}}}}{\verdadero \text{ \faIcon{glass-cheers}}}
        $$

        Entonces cambio ese $\land$ por un $\entonces_L$, donde me aseguro de lidiar con potenciales
        indefiniciones de $P(i)$ que harían que todo explote:
        $$
          (\paratodo i : \enteros)\ub{(\ob{(0 \leq i < 10)}{f(i)} \entoncesL P(i))}{S(i)} \sii (\paratodo i : \enteros)(S(i))
        $$
        Ahora la situación es más feliz:
        $$
          (\paratodo i : \enteros)(S(i))
          \sii
          S(-N) \land S(-N+1)
          \land \dots \land
          \blue{S(0)}
          \land \blue{\dots} \land
          \blue{S(9)} \land
          S(10)
          \land \dots \land
          S(N)
        $$
        Todos los $\blue{S(i)}$ son verdaderos y no se indefinen por el $\red{\entoncesL}$

  \item "\textit{Algún natural menor a 10 cumple P}":
        $$
          (\existe i : \enteros)((0 \leq i < 10) \entonces P(i))
        $$

  \item "\textit{Todos los naturales menores a 10 que cumplen P y Q}":
        $$
          (\paratodo x : \enteros)((0 \leq x < 10) \entonces (P(x) \land Q(x)))
        $$

  \item "\textit{No hay ningún natural menor a 10 que cumpla $P$ y $Q$}":
        $$
          (\paratodo x : \enteros)((0 \leq x < 10) \entonces (P(x) \land Q(x)))
        $$
        Propongo
        $$
          (\paratodo x : \enteros)\bigl((\ub{0 \leq x < 10}{f(x)}) \red{\entoncesL} \ub{
          \red{\neg}(P(x) \red{\andL} Q(x))} {g(x)}\bigr)
        $$
        Con el $\entoncesL$ mato los casos que no cumplen $f(x)$. Luego $g(x)$ solo toma valor \falso cuando $P(x)$ y $Q(x)$ son ambas \verdadero.
\end{enumerate}
